\documentclass[12pt]{report}
\usepackage{minted}
\usepackage{verbatim}
\usepackage{fullpage}
\usepackage{etoolbox}
\usepackage{lipsum}
\usepackage{graphicx}
\usepackage{hyperref}




\graphicspath{ {res/} }

\newcommand{\HRule}{\rule{\linewidth}{0.5mm}} 
\renewcommand\emph{\textbf}
\renewcommand{\baselinestretch}{1.1} 

\begin{document}
    \begin{titlepage}

        \center

        \textsc{\LARGE Universita' degli Studi di Messina}\\[0.1cm] 
        \textsc{\Large Dipartimento di Matematica e Informatica}\\[0.5cm] 
        \textsc{\Large Database course project}\\[0.5cm] 

        \HRule \\[0.4cm]
        { \huge \bfseries veeForum}\\[0.1cm]

        {\large 23 March 2015}
        \HRule \\[1.5cm]
         
        \begin{minipage}{0.4\textwidth}
        \begin{flushleft} \large
        \emph{Author:}\\
        Vittorio \textsc{Romeo} % Your name
        \end{flushleft}
        \end{minipage}
        ~
        \begin{minipage}{0.4\textwidth}
        \begin{flushright} \large
        \emph{Professor:} \\
        Massimo \textsc{Villari}

        
        \end{flushright}
        \end{minipage}\\[4cm]

        \vfill

        \begin{minipage}{\linewidth}
            \centering
            \begin{minipage}{0.35\linewidth}
                \begin{figure}[H]
                    \center
                    \includegraphics[width=2cm, height=2cm]{logovee}
                    
                    http://vittorioromeo.info
                \end{figure}
            \end{minipage}
            \hspace{0.27\linewidth}
            \begin{minipage}{0.35\linewidth}
                \begin{figure}[H]
                    \center
                    \includegraphics[width=2cm, height=2cm]{logounime}
                    
                    http://unime.it
                \end{figure}
            \end{minipage}
        \end{minipage}\\[3cm]
    \end{titlepage}

    \pagenumbering{gobble}
    \newcommand{\atoc}[1]{\addtocontents{toc}{#1\par}}
    \renewcommand{\thesection}{\arabic{section}.}
    \tableofcontents
    \newpage
    \pagenumbering{arabic}

    \part{Non-technical}
        aaa

        \chapter{Client request}
            The client requests the design and implementation of a \emph{forum creation/management framework} and a \emph{modern responsive web forum browsing/management application}.

            The client intends using the requested forum framework \emph{to build communication platforms} for various projects, both for internal employee usage and interaction with the public.

            It is imperative for the system to allow administrators to easily well-organized create \emph{content-section hierarchies} and \emph{user-group hierarchies}.

            Administrators also need to be able to \emph{give groups specific permissions for every section}.

            Some sections will only be visible and editable to employee groups (e.g. internal discussion), some sections will be visible but not editable by the public (e.g. announcements), and others will need to be completely open to the public (e.g. technical support).

            Being able to \emph{keep track of user-created content} is also very important for the client.

            Initially, tracking the date and the author of the content will be enough, but the system has to be designed in such a way that inserting additional creation information (e.g. browser/operating system used to post) will be trivial. 

            In the future, additional content types (e.g. videos, attachments) may be added to the system and their creation will have to be tracked as well.

            This data needs to be independent from the contents, in order to easily allow administrators and project managers to gather statistical data on forum usage.

            The web application has to be extremely simple but flexible as well. Administrators need be able to perform all functions described above through a responsive admin panel.

            Content consumers and creators should be able to view and create content from the same responsive interface.

            Moderators and administrators should be able to edit and delete posts through the same interface as well. User interface controls will be shown/hidden depending on the user’s permissions.

        \chapter{Software Requirements Specification}
            aaa

            \section{Introduction}
                This \emph{Software Requirements Specification} (SRS) chapter contains all the information needed by software engineers and project managers to design and implement the requested forum creation/management framework.

                The SRS was written following the \emph{Institute of Electrical and Electronics Engineers} (IEEE) guidelines on SRS creation.



                \subsection{Purpose}
                    The SRS chapter is contained in the \emph{non-technical} part of the thesis.
                    
                    Its purpose is providing a \emph{comprehensive description} of the objective and environment for the software under development.

                    The SRS fully describes \emph{what the software will do} and \emph{how it will be expected to perform}.



                \subsection{Scope}

                    \subsubsection{Identity}
                        The software that will be designed and produced will be called \emph{veeForum}.



                    \subsubsection{Feature extents}

                        The complete product will:

                        \begin{itemize}
                            \item Provide a framework for the \emph{creation and the management of a forum system}.
                            \item Allow its users to \emph{deploy and administrate} multi-purpose forums.
                            \item Give access to a \emph{modern responsive web application} to setup, browse and manage the forum.
                        \end{itemize}

                        veeForum, however, will not:

                        \begin{itemize}
                            \item Provide infrastructure or implementation for a complete blog/website. The scope of the software is forum building.
                            \item Implement instant private messaging - user-to-user chat is beyond the scope of the project.
                        \end{itemize}



                    \subsubsection{Benefits and objectives}

                        Deploying veeForum will give its users a number of important benefits and will fulfill specific objectives.

                        \begin{itemize}
                            \item Companies and individuals making use of veeForum will have access to an \emph{easy-to-install} and \emph{easy-to-use} forum creation and management platform.
                            \item Users and moderators of the deployed forums will be able to \emph{easily create, track and manage} content and other forum users.
                            \item Forum administrators will be given \emph{total control} of the forum structure, users and permissions through an \emph{easy-to-use} responsive administration panel. 
                        \end{itemize}

                \subsection{Definitions, acronyms and abbreviations}
                    aaa

                \subsection{Overview}
                    aaa

            \section{General description}
                aaa

                \subsection{Product perspective and functions}
                    The product shares many basic aspects and features with existing forum frameworks such as \emph{phpBB} or \emph{vBulletin}: flat/threaded discussion support, nested sections, user attachments, etc.

                    veeForum improves on existing forum frameworks in the following ways:

                    \begin{itemize}
                        \item Provides a responsive web interface without postbacks.
                        \item Allows users and moderators to subscribe and unsubscribe not only to posts, but to users and sections as well.
                        \item Has a powerful real-time Facebook-like notification system that notifies users when tracked content has been added or edited.
                        \item Gives administrator the possibility to design and manage complex permission hierarchies for user groups and single users.
                    \end{itemize}

                \subsection{User characteristics}
                    veeForum needs to target both users that \emph{only consume the content offered by deployed forums}, users that \emph{actively create and manage content in deployed forums}, and users that \emph{build and deploy forum instances}.

                    User-friendliness is essential for every target, but all the required functionality is effectively exposed to different user groups.

                    It is therefore required to have clear interfaces thar do not negatively affect the user experience by being either too complex or too simple (all features need to be exposed).

            \section{Specific requirements}
                aaa

                \subsection{External interface requirements}
                    aaa

                    \subsubsection{User interfaces}
                        The product will provide both a desktop and a mobile user web interface.

                        \begin{itemize}
                            \item \emph{Web interface}: it is required to provide a modern responsive web interface, compatible and tested with the most popular browsers (Internet Explorer 10+, Google Chrome, Mozilla Firefox). The web interface will give forum access to users and moderators, and administrator access to forum management staff.
                            \item \emph{Mobile interface}: is is required to provide a modern mobile application for the major platforms (Android, iOS, Windows Phone). The mobile application will allow browsing and content management of forums created with the product. 
                        \end{itemize}

                    \subsubsection{Software interfaces}
                        The \emph{open-source policy} of veeForum will allow framework users to expand or improve existing functionality and to interact with other existing technologies.

                        Interface

        \chapter{Glossary}
            aaa

        \chapter{Usage}
            aaa

    \part{Technical}
        aaa

        \chapter{Development}
            aaa

            \section{Environment}
                aaa

            \section{Docker}
                aaa

            \section{Version control}
                aaa

            \section{Thesis}
                aaa

                \subsection{\LaTeX}
                    aaa

                \subsection{LatexPP}
                    aaa

        \chapter{Project structure}
            aaa

            \section{Folder structure}
                aaa

            \section{PHP Module}
                aaa

            \section{SQL Module}
                aaa

            \section{Other data}
                aaa

        \chapter{Installation}
            aaa

        \chapter{Conceptual model}
            aaa

        \chapter{Logical model}
            aaa

        \chapter{Table details} 
            aaa

        \chapter{Web interface}
            aaa
            
        \chapter{Sample queries}
            aaa

    \part{Conclusion}
        aaa

        \chapter{Final product}
            aaa

        \chapter{What I learned}
            aaa

        \chapter{Future}
            aaa

        \chapter{References}
            aaa

\end{document}
