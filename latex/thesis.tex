\documentclass[12pt]{report}
\usepackage{verbatim}
\usepackage{fullpage}
\usepackage{etoolbox}
\usepackage{lipsum}
\usepackage{graphicx}
\usepackage{hyperref}
\usepackage{titlesec}
\usepackage{pdftexcmds} 
\usepackage{minted}





\graphicspath{ {res/} }
\setcounter{secnumdepth}{4}
\titleformat{\paragraph}
{\normalfont\normalsize\bfseries}{\theparagraph}{1em}{}
\titlespacing*{\paragraph}
{0pt}{3.25ex plus 1ex minus .2ex}{1.5ex plus .2ex}

\newcommand{\HRule}{\rule{\linewidth}{0.5mm}}
\renewcommand\emph{\textbf}
\renewcommand{\baselinestretch}{1.1}

\newcommand{\printSQLtest}[1]
{
    \inputminted[linenos, breaklines, breakbytoken, tabsize=4, fontsize=\footnotesize]{mysql}{#1}
}

\newcommand{\printSQLTablepage}[2]
{    
    \subsection{#2}
    \subsubsection{Code}
    \printSQLtest{../sql/parts/#1}
    \subsubsection{Explanation}
    \newpage
}

\begin{document}
    \begin{titlepage}

        \center

        \textsc{\LARGE Universita' degli Studi di Messina}\\[0.1cm]
        \textsc{\Large Dipartimento di Matematica e Informatica}\\[0.5cm]
        \textsc{\Large Database course project}\\[0.5cm]

        \HRule \\[0.4cm]
        { \huge \bfseries veeForum}\\[0.1cm]

        {\large 29 May 2015}
        \HRule \\[1.5cm]

        \begin{minipage}{0.4\textwidth}
        \begin{flushleft} \large
        \emph{Author:}\\
        Vittorio \textsc{Romeo}
        \end{flushleft}
        \end{minipage}
        ~
        \begin{minipage}{0.4\textwidth}
        \begin{flushright} \large
        \emph{Professor:} \\
        Massimo \textsc{Villari}


        \end{flushright}
        \end{minipage}\\[4cm]

        \vfill

        \begin{minipage}{\linewidth}
            \centering
            \begin{minipage}{0.35\linewidth}
                \begin{figure}[H]
                    \center
                    \includegraphics[width=2cm, height=2cm]{logovee}

                    http://vittorioromeo.info
                \end{figure}
            \end{minipage}
            \hspace{0.27\linewidth}
            \begin{minipage}{0.35\linewidth}
                \begin{figure}[H]
                    \center
                    \includegraphics[width=2cm, height=2cm]{logounime}

                    http://unime.it
                \end{figure}
            \end{minipage}
        \end{minipage}\\[3cm]
    \end{titlepage}

    \pagenumbering{gobble}
    \newcommand{\atoc}[1]{\addtocontents{toc}{#1\par}}
    \renewcommand{\thesection}{\arabic{section}.}
    \tableofcontents
    \newpage
    \pagenumbering{arabic}

    \part{Project specifications}
        The following part of the document describes the project and its design/development process without exploring its implementation details.

        The part begins with a synthesis of the \emph{client request}. After a careful analysis of the request, a \emph{Software Requirements Specification} (SRS) was written.

        Writing a correct and informative SRS is of utmost importance to achieve an high-quality final product and ensuring the development process goes smoothly.

        The SRS will cover the following points in depth:

        \begin{itemize}
            \item \emph{Scope and purpose}. 
            \item \emph{Feature and functions}. 
            \item \emph{External interface requirements}.
            \item \emph{Functional requirements}.
            \item \emph{Example use cases}.
            \item \emph{Non-functional requirements}.
            \item \emph{Analysis models}.
        \end{itemize}

        \chapter{Client request}

            The client requests the design and implementation of a \emph{forum creation/management framework} and a \emph{modern responsive web forum browsing/management application}.

            The client intends using the requested forum framework \emph{to build communication platforms} for various projects, both for internal employee usage and interaction with the public.

            It is imperative for the system to allow administrators to easily well-organized create \emph{content-section hierarchies} and \emph{user-group hierarchies}.

            Administrators also need to be able to \emph{give groups specific permissions for every section}.

            Some sections will only be visible and editable to employee groups (e.g. internal discussion), some sections will be visible but not editable by the public (e.g. announcements), and others will need to be completely open to the public (e.g. technical support).

            Being able to \emph{keep track of user-created content} is also very important for the client.

            Initially, tracking the date and the author of the content will be enough, but the system has to be designed in such a way that inserting additional creation information (e.g. browser/operating system used to post) will be trivial.

            In the future, additional content types (e.g. videos, attachments) may be added to the system and their creation will have to be tracked as well.

            Users and moderators will also need to be able to track user content through a \emph{real-time notification system} directly from the web application interface.

            This data needs to be independent from the contents, in order to easily allow administrators and project managers to gather statistical data on forum usage.

            The web application has to be extremely simple but flexible as well. Administrators need be able to perform all functions described above through a responsive admin panel.

            Content consumers and creators should be able to view and create content from the same responsive interface.

            Moderators and administrators should be able to edit and delete posts through the same interface as well. User interface controls will be shown/hidden depending on the user’s permissions.

        \chapter{Software Requirements Specification}
            \section{Introduction}
                This \emph{Software Requirements Specification} (SRS) chapter contains all the information needed by software engineers and project managers to design and implement the requested forum creation/management framework.

                The SRS was written following the \emph{Institute of Electrical and Electronics Engineers} (IEEE) guidelines on SRS creation.

                \subsection{Purpose}
                    The SRS chapter is contained in the \emph{non-technical} part of the thesis.

                    Its purpose is providing a \emph{comprehensive description} of the objective and environment for the software under development.

                    The SRS fully describes \emph{what the software will do} and \emph{how it will be expected to perform}.



                \subsection{Scope}

                    \subsubsection{Identity}
                        The software that will be designed and produced will be called \emph{veeForum}.

                    \subsubsection{Feature extents}

                        The complete product will:

                        \begin{itemize}
                            \item Provide a framework for the \emph{creation and the management of a forum system}.
                            \item Allow its users to \emph{deploy and administrate} multi-purpose forums.
                            \item Give access to a \emph{modern responsive web application} to setup, browse and manage the forum.
                        \end{itemize}

                        veeForum, however, will not:

                        \begin{itemize}
                            \item Provide infrastructure or implementation for a complete blog/website. The scope of the software is forum building.
                            \item Implement instant private messaging - user-to-user chat is beyond the scope of the project.
                        \end{itemize}

                    \subsubsection{Benefits and objectives}

                        Deploying veeForum will give its users a number of important benefits and will fulfill specific objectives.

                        \begin{itemize}
                            \item Companies and individuals making use of veeForum will have access to an \emph{easy-to-install} and \emph{easy-to-use} forum creation and management platform.
                            \item Users and moderators of the deployed forums will be able to \emph{easily create, track and manage} content and other forum users.
                            \item Forum administrators will be given \emph{total control} of the forum structure, users and permissions through an \emph{easy-to-use} responsive administration panel.
                        \end{itemize}
            
            \section{General description}                            
                \subsection{Product perspective and functions}
                    The product shares many basic aspects and features with existing forum frameworks such as \emph{phpBB} or \emph{vBulletin}: flat/threaded discussion support, nested sections, user attachments, etc.

                    veeForum improves on existing forum frameworks in the following ways:

                    \begin{itemize}
                        \item Provides a responsive web interface without postbacks.
                        \item Allows users and moderators to subscribe and unsubscribe not only to posts, but to users and sections as well.
                        \item Has a powerful real-time Facebook-like notification system that notifies users when tracked content has been added or edited.
                        \item Gives administrator the possibility to design and manage complex permission hierarchies for user groups and single users.
                    \end{itemize}

                \subsection{User characteristics}
                    veeForum needs to target both users that \emph{only consume the content offered by deployed forums}, users that \emph{actively create and manage content in deployed forums}, and users that \emph{build and deploy forum instances}.

                    User-friendliness is essential for every target, but all the required functionality is effectively exposed to different user groups.

                    It is therefore required to have clear interfaces thar do not negatively affect the user experience by being either too complex or too simple (all features need to be exposed).

            \section{Specific requirements}

                \subsection{External interface requirements}

                    \emph{External interface requirements} identify and document the interfaces to other systems and external entities within the project scope. 

                    \subsubsection{User interfaces}
                        The product will provide both a desktop and a mobile user web interface.

                        \begin{itemize}
                            \item \emph{Web interface}: it is required to provide a modern responsive web interface, compatible and tested with the most popular browsers (Internet Explorer 10+, Google Chrome, Mozilla Firefox). The web interface will give forum access to users and moderators, and administrator access to forum management staff.
                            \item \emph{Mobile interface}: is is required to provide a modern mobile application for the major platforms (Android, iOS, Windows Phone). The mobile application will allow browsing and content management of forums created with the product.
                        \end{itemize}

                    \subsubsection{Software interfaces}
                        The \emph{open-source policy} of veeForum will allow framework users to expand or improve existing functionality and to interact with other existing technologies.

                        Accessing and modifying forum data (assuming permission requirements are satisfied by the user) will be possible through \emph{RESTful} requests, returning and accepting \emph{JSON} (Javascript Object Notation).

                \subsection{Functional requirements}
                    In software engineering, a \emph{functional requirement} defines a function of a system and its components. 

                    Functional requirements may be \emph{calculations}, \emph{technical details}, \emph{data manipulation and processing} and other specific functionality that define what a system is supposed to accomplish. 

                    Behavioral requirements describing all the cases where the system uses the functional requirements are captured in \emph{use cases}.

                    \subsubsection{User/group management}
                        \begin{itemize}
                            \item \emph{Users}: users will be managed by the system. Users can register (or be manually added by an administrator). Registration can be configured to require a confirmation email or not.
                            \item \emph{Groups}: every user will be part of at least one group at all times. Groups are part of an hierarchy: they can inherit from each other. Groups can have permissions specific to sections and system-wide permissions.
                        \end{itemize}

                     \subsubsection{Content hierarchy}
                        \begin{itemize}
                            \item \emph{Posts}: posts will be the base of the content hierarchy. They will contain HTML-enabled text and any number of attachments.
                            Posts can be edited and deleted by the original owner.
                            \item \emph{Threads}: threads are groups of posts. Users with the correct permissions can create a thread in a specific section and have other users add posts or subscribe to it.
                            Threads can be edited and deleted by the original owner.
                            \item \emph{Sections}: sections are content containers intended to group threads related to the same subject. Forum administrators and moderators can create sections and give users permissions to view or edit them.
                        \end{itemize}

                      \subsubsection{Content tracking system}
                        \begin{itemize}
                            \item \emph{Creation data}: user-created content (posts, threads,attachments, etc) will have some data specific to its creation can be extended by forum administrators. Basic predefined data will consist of creation date and time. It will be possible to run statistical queries on content creation data.
                            \item \emph{Subscriptions}: users and moderators will be able to subscribe to specific sections, threads or user to track their contents. They will receive real-time notifications upon addition/editing of tracked content.
                        \end{itemize}

                \subsection{Example use cases}
                    In software and systems engineering, a \emph{use case} is a list of steps, typically defining interactions between one or more actors and a system, to achieve a goal. 

                    \subsubsection{Mobile game forum}
                        A company developed a popular mobile game, with a wide audience.
                        The company uses the \emph{veeForum framework} to give users a place to discuss game strategy, give feedback on the quality of their product and receive technical support.

                        \paragraph{Actors}
                            \begin{itemize}
                                \item Game developers.
                                \item Game players.
                                \item Forum management team.
                                \item Technical support team.
                                \item Feedback (PR) team.
                            \end{itemize}

                        \paragraph{Pre-conditions}
                            \begin{itemize}
                                \item Release of a popular product with a wide audience.                            
                                \item Game users need to register on the forum.
                            \end{itemize}

                        \paragraph{Flow of events}
                            \begin{itemize}
                                \item Installation and configuration of a veeForum-enabled forum system by the forum management team.                            
                                \item Creation of the sections and permission hierarchies by the forum management team.
                                \item Registration and content creation by the game developers and game players.
                            \end{itemize}

                        \paragraph{Post-conditions}
                            \begin{itemize}
                                \item Game players will be able to share their strategies and thoughts on the product.
                                \item The technical support team will find all technical issues grouped in a convenient way and will be able to track individual issues. Technical support members will be able to communicate with each other in a private section.
                                \item The feedback team will be able to track user suggestions and forward potential product improvements to the developer team.
                            \end{itemize}

                    \subsubsection{Local city GNU/Linux usergroup forum}
                        Some GNU/Linux users from the same city decide to start a local usergroup to discuss the GNU/Linux ecosystem and make new friends. In spirit with the open-source nature of the system, collaboration is extremely important. They require to easily assign specific permissions to users and groups to allow the forum to grow and be well-organized.

                        \paragraph{Actors}
                            \begin{itemize}
                                \item Usergroup creators.
                                \item Usergroup members.
                                \item External visitors.
                            \end{itemize}

                        \paragraph{Pre-conditions}
                            \begin{itemize}
                                \item Interest in a local GNU/Linux usergroup.                            
                                \item Availability of people willing to collaborate.
                            \end{itemize}

                        \paragraph{Flow of events}
                            \begin{itemize}
                                \item Installation and configuration of a veeForum-enabled forum system by the usergroup creators.
                                \item Creation of the initial sections and permission hierarchies by the usergroup creators.
                                \item Registration of usergroup members and external visitors.
                                \item The usergroup creators give other usergroup members permissions to create and manage sections and users, starting a chain of collaborative forum content development.
                                \item Usergroup members and external visitors contribute and make use of the content.
                            \end{itemize}

                        \paragraph{Post-conditions}
                            \begin{itemize}
                                \item Local city usergroup members will be able to get to know and speak to each other.
                                \item Usergroups members willing to contribute will be able to easily manage sections and write posts/articles.
                                \item External visitors will be able to make use of the public content.
                            \end{itemize}

                \subsection{Non-functional requirements}   
                    Functional requirements are supported by \emph{non-functional requirements} (also known as quality requirements), which impose constraints on the design or implementation (such as performance requirements, security, or reliability). 

                    \subsubsection{Performance}
                        The system will be designed from the ground-up with emphasis on performance. As the forum may have huge amounts of contents and concurrent usage after its deployment, optimizing is a must.

                        When possible, functions will be implemented \emph{directly in the database}, for maximum performance.

                        Web backend functions will also be carefully \emph{optimized both for memory and speed}.

                    \subsubsection{Reliability}
                        The system will have to be reliable and keep working in case of errors. 

                        Database queries and functions will be executed in \emph{safe wrappers} that catch and handle errors carefully.

                    \subsubsection{Security}
                        veeForum needs to guarantee privacy and security for users and administrator of the system.

                        Well-tested and well-received \emph{security idioms} and \emph{encryption algorithms} will have to be used throughout the implementation of the whole system.

                    \subsubsection{Maintainability and portability}
                        Being an open-source project, \emph{maintainability}, \emph{extensibility} and \emph{portability} are key.

                        The code layer will be carefully designed and organized to allow easy maintenance, bugfixing and feature addition.

                        To ensure maximum portability, the product will be designed to work on the most popular \emph{GNU/Linux} distributions and will be thoroughly tested on different platforms.

                \subsection{Logical Database Requirements}
                    aaa

            \section{Analysis models}
                aaa
                
                \subsection{Sequence Diagrams}
                    aaa

                \subsection{Data Flow Diagrams}
                    aaa

                \subsection{State-Transition Diagrams}
                    aaa

    \part{Technical analysis}
        The following part of the thesis will cover all implementation choices and details for veeForum in depth.

        Firstly, the \emph{development environment and tools} and \emph{chosen technologies} will be described and motivated.

        Afterwards, the technical details, including code examples and APIs, will be described for the two modules of the application: the \emph{database} and the \emph{web application}.

        Every \emph{table} of the database will be analyzed in detail, directly showing commented \emph{DDL} code. The database also contains important \emph{stored procedures} and \emph{triggers} that are core part of the system's logic and that need to be explained in depth - the related \emph{DML} code will be shown and commented.

        The web application itself is divided in multiple modules:
        \begin{itemize}
            \item A \emph{database interface backend module}, that interfaces with the database and wraps its tables and stored procedures.
            \item A \emph{HTML5 generation module}, that greatly simplifies the creation of dynamic forum web pages by wrapping HTML5 controls in \emph{object-oriented wrappers} that can be easily bound to callbacks and database events.
            \item A \emph{modern responsive AJAX frontend} that allows users and interact with the backend module from multiple device, limiting postbacks and page refreshes.
        \end{itemize}
         
        \chapter{Development process}

            \section{Environment and tools}
                All modules of veeForum have been developed on \emph{Arch Linux x64}, a lightweight GNU/Linux distribution.

                Arch is installed as a minimal base system, configured by the user upon which their own ideal environment is assembled by installing only what is required or desired for their unique purposes. GUI configuration utilities are not officially provided, and most system configuration is performed from the shell and a text editor. Based on a rolling-release model, Arch strives to stay bleeding edge, and typically offers the latest stable versions of most software.

                No particular integrated development environments (IDEs) were used during the development - a modern graphical text editor, \emph{Sublime Text 3}, was used instead.

            \section{Docker}
                Docker is an open-source project that \emph{automates the deployment of applications} inside software containers, by providing an additional layer of abstraction and automation of operating-system-level virtualization on Linux.

                Docker uses resource isolation features of the Linux kernel such as \emph{cgroups} and \emph{kernel namespaces} to allow independent containers to run within a single Linux instance.

                This technology has been used since the beginning of the development process to \emph{separate veeForum data and packages} from the host system and to dramatically increase \emph{portability} and \emph{ease of testing}.

                Docker is also used for the installation of the product on target systems - with a single command it is possible to \emph{retrieve all required dependencies}, correctly \emph{configure the system} and \emph{automatically install veeForum}.

            \section{Version control system}
                Version control systems (VCSs) allow the \emph{management of changes} to documents, computer programs, large web sites, and other collections of information.

                Nowadays, a version control system is \emph{essential} for the development of any project. 
                Being able to track changes, develop features in separate \emph{branches}, have multiple programmers work on the same code base without conflicts and much more is extremely important for projects of any scope and size.

                The chosen VCS is \emph{Git}, a distributed revision control system with an emphasis on \emph{speed}, \emph{data integrity}, and support for \emph{distributed, non-linear workflows}.

                Git is widely appreciated in the private and open-source programming communities - it was initially designed and developed by \emph{Linus Torvalds} for Linux kernel development in 2005, and has since become the most widely adopted version control system for software development.

                The veeForum project is \emph{open-source} and \emph{appreciates feedback and contributions}. It is hosted on \emph{GitHub}, a web-based Git repository hosting service, which offers all of the distributed revision control and source code management (SCM) functionality of Git, while adding \emph{additional features} that make collaboration and public contributions easy and accessible.

            \section{Apache}
                The Apache HTTP server is the world's most widely used web server software. 

                Apache has been under open-source development for about 20 years - it supports all modern server-side technologies and programming languages, and also is \emph{extremely reliable} and \emph{secure}.
                        
            \section{Thesis}
                The current document was written using \LaTeX, an high-quality typesetting system; it includes features designed for the production of \emph{technical and scientific documentation}.

                \LaTeX was chosen for the current document because of the visually pleasant typography, its extensibility features and its abilities to include and highlight source code.

                \subsection{LatexPP}
                    A small \emph{C++14} \LaTeX preprocessor named \emph{LatexPP} was developed for the composition of this thesis.

                    LatexPP allows to use an intuitive syntax that avoids markup repetition for code highlighting and macros.

                    Preprocessing and compiling a \LaTeX document using LatexPP is simple and can be automated using a simple \emph{bash} script.

\begin{minted}[mathescape, linenos, numbersep=5pt, gobble=2, frame=lines, framesep=2mm]{bash}
    #!/bin/bash

    latexpp ./thesis.lpp > ./thesis.tex
    pdflatex -shell-escape ./thesis.tex && chromium ./thesis.pdf    
\end{minted}

                    LatexPP is available as an open-source project on GitHub:

                    \url{https://github.com/SuperV1234/Experiments/Random}


        \chapter{Project structure}
            The project folder and file structure is organized as such:

            \begin{itemize}
                \item \emph{./doc/}
                    
                    Folder containing the documentation of the project.
                    \begin{itemize}
                        \item \emph{./latex/} 

                        LatexPP and \LaTeX{}  source and output files.                            
                    \end{itemize}
            
                \item \emph{./sql/}

                    Folder containing the SQL DDL scripts.
                    \begin{itemize}
                        \item \emph{./scripts/}

                        Contains all the parts that make up the complete SQL initialization script.
                        
                        \item \emph{./mkScript.sh}

                        Builds the complete SQL initialization scripts from the files in ./scripts/.

                        \item \emph{./script.sql}

                        Complete SQL initialization scripts that sets up a database suitable veeForum.
                    \end{itemize}

                \item \emph{./exe/}
                    
                    Folder containing executable scripts to setup the system.
                    \begin{itemize}
                        \item \emph{./docker/} 

                        Docker-related scripts.
                        \begin{itemize}
                            \item \emph{./start.sh} 

                            Starts a Docker instance containing veeForum.

                            \item \emph{./cleanup.sh} 

                            Cleans any running veeForum Docker instance.

                            \item \emph{./shell.sh} 

                            Starts a Docker instance containing veeForum, controlling an instance of bash inside it.

                            \item \emph{./httpdLog.sh} 

                            Prints the Apache error log of the current running veeForum Docker instance.
                        \end{itemize}

                    \end{itemize}

                \item \emph{./www/}
                    
                    Folder containing web application data.
                    \begin{itemize}

                        \item \emph{./css/} 

                        CSS3 stylesheets.

                        \item \emph{./js/} 

                        ECMAScript 5 script files.

                        \item \emph{./json/} 

                        Non-relational data storage files, in JSON format.

                        \item \emph{./php/} 

                        PHP backend code. 
                        \begin{itemize}
                            \item \emph{./lib/} 

                            Backend to database interface library and HTML5 generation library.

                            \item \emph{./core/} 

                            PHP frontend files that generate the responsive HTML5 web application user interface.
                        \end{itemize}

                    \end{itemize}

            \end{itemize}

            \section{PHP Module}
                aaa

            \section{SQL Module}
                aaa

            \section{Other data}
                aaa

        \chapter{SQL}

            \newpage
            \section{Database setup}

                veeForum is supposed to be installed on a clean instance of MySQL server. The following script correctly initializes the required database and cleans any previous version of veeForum.
                
                \printSQLTablepage{00_db.sql}{db}
                    This script is meant to be run once to create and initialize from scratch the whole MySQL veeForum backend.
                    Therefore, we drop the database if exists and re-create it.

            \newpage
            \section{Tables}

                A big amount of tables is required to make veeForum satisfy all requirements. Every table in the project is documented in the following section - the full \emph{DDL} commented code and an explanation is provided for every table.

                \printSQLTablepage{01_tblLog.sql}{log}
                    The \emph{log} table is a simple non-relational list of log messages that can be used for debugging and security purposes. 

                \printSQLTablepage{02_tblTag.sql}{tag}
                    The \emph{tag} table is a simple non-relational list of unique tags that can be attached to user-created content.

                \printSQLTablepage{03_tblGroup.sql}{group}
                    The \emph{group} table defines the groups users can belong to. Every row defines a different group and assigns forum-wide permissions to them.
                    Groups can inherit from each other thanks to the \mintinline{sql}{id_parent} field, which is the id of the parent group and can be \mintinline{sql}{NULL}.

                    \begin{figure}[h]
                    \caption{Group parenthood relationship.}
                    \centering
                    \includegraphics[width=0.5\textwidth]{td/03group}
                    \end{figure}

                \printSQLTablepage{04_tblUser.sql}{user}
                    The \emph{group} table contains the users registered to the forum system. Every user \emph{needs} to belong to a group, whose id is stored in \mintinline{sql}{id_group}.
                    Every row stores user credentials data and personal info.

                    \begin{figure}[h]
                    \caption{User-group relationship.}
                    \centering
                    \includegraphics[width=0.5\textwidth]{td/04user}
                    \end{figure}

                \printSQLTablepage{05_tblSection.sql}{section}
                    The \emph{section} table contains all forum sections, defining the base hierarchy for content. Sections have a name and can inherit from each other thanks to the \mintinline{sql}{id_parent} field, which is the id of the parent section and can be \mintinline{sql}{NULL}.

                    \begin{figure}[h]
                    \caption{Section parenthood relationship.}
                    \centering
                    \includegraphics[width=0.5\textwidth]{td/05section}
                    \end{figure}

                \printSQLTablepage{06_tblFileData.sql}{fileData}
                    The \emph{fileData} table stores binary data and a filename for attachments. It makes use of the \mintinline{sql}{blob} MySQL data type to directly store binary data in the database backend.

                \printSQLTablepage{07_tblContentBase.sql}{contentBase}
                    \lipsum[1-1]

                    \begin{figure}[h]
                    \caption{Content specializations.}
                    \centering
                    \includegraphics[width=0.5\textwidth]{td/07contentbase}
                    \end{figure}

                \printSQLTablepage{08_tblContentThread.sql}{contentThread}
                    \lipsum[1-1]


                \printSQLTablepage{09_tblContentPost.sql}{contentPost}
                    \lipsum[1-1]


                \printSQLTablepage{10_tblContentAttachment.sql}{contentAttachment}
                    \lipsum[1-1]


                \printSQLTablepage{11_tblSubscriptionBase.sql}{subscriptionBase}
                    \lipsum[1-1]

                    \begin{figure}[h]
                    \caption{Subscription specializations.}
                    \centering
                    \includegraphics[width=0.5\textwidth]{td/11subscriptionbase}
                    \end{figure}


                \printSQLTablepage{12_tblSubscriptionThread.sql}{subscriptionThread}
                    \lipsum[1-1]


                \printSQLTablepage{13_tblSubscriptionUser.sql}{subscriptionUser}
                    \lipsum[1-1]


                \printSQLTablepage{14_tblSubscriptionTag.sql}{subscriptionTag}
                    \lipsum[1-1]


                \printSQLTablepage{15_tblNotificationBase.sql}{notificationBase}
                    \lipsum[1-1]

                    \begin{figure}[h]
                    \caption{Notification specializations.}
                    \centering
                    \includegraphics[width=0.5\textwidth]{td/11subscriptionbase}
                    \end{figure}


                \printSQLTablepage{16_tblNotificationUser.sql}{notificationUser}
                    \lipsum[1-1]


                \printSQLTablepage{17_tblNotificationThread.sql}{notificationThread}
                    \lipsum[1-1]


                \printSQLTablepage{18_tblNotificationTag.sql}{notificationTag}
                    \lipsum[1-1]


                \printSQLTablepage{19_tblTagContent.sql}{tagContent}
                    \lipsum[1-1]

                    \begin{figure}[h]
                    \caption{Tag-content relationship.}
                    \centering
                    \includegraphics[width=0.5\textwidth]{td/19tagcontent}
                    \end{figure}


                \printSQLTablepage{20_tblGroupSectionPermission.sql}{groupSectionPermission}
                    \lipsum[1-1]

            \newpage
            \section{Stored procedures}

                \printSQLTablepage{21_procsMkContent.sql}{mkContent}
                    \lipsum[1-1]


                \printSQLTablepage{22_procsMkSubscription.sql}{mkSubscription}
                    \lipsum[1-1]


                \printSQLTablepage{23_procsMkNotification.sql}{mkNotification}
                    \lipsum[1-1]


                \printSQLTablepage{24_procsUtils.sql}{utils}
                    \lipsum[1-1]


                \printSQLTablepage{25_procGNUser.sql}{gNUser}
                    \lipsum[1-1]


                \printSQLTablepage{26_procGNThread.sql}{gNThread}
                    \lipsum[1-1]


                \printSQLTablepage{27_procGNTag.sql}{gNTag}
                    \lipsum[1-1]


                \printSQLTablepage{28_procCalcPrivs.sql}{calcPrivs}
                    \lipsum[1-1]


                \printSQLTablepage{29_procCalcPerms.sql}{calcPerms}
                    \lipsum[1-1]

            \newpage
            \section{Triggers}

                \printSQLTablepage{90_trgNotifications.sql}{notifications}
                    \lipsum[1-1]


                \printSQLTablepage{91_trgDelContentBase.sql}{contentBase}
                    \lipsum[1-1]


                \printSQLTablepage{92_trgDelSubscriptionBase.sql}{subscriptionBase}
                    \lipsum[1-1]


                \printSQLTablepage{93_trgDelNotificationBase.sql}{notificationBase}
                    \lipsum[1-1]


                \printSQLTablepage{94_trgDelSubscriptionNtf.sql}{subscriptionNtf}
                    \lipsum[1-1]


                \printSQLTablepage{95_trgDelSubCnt.sql}{delSubCnt}
                    \lipsum[1-1]

            \newpage
            \section{Database inizialization}

                \printSQLTablepage{99_initialize.sql}{initialize}
                    \lipsum[1-1]



        \chapter{Installation}
            aaa

        \chapter{Conceptual model}
            aaa

        \chapter{Logical model}
            aaa

        \chapter{Table details}
            aaa

        \chapter{Web interface}
            aaa

        \chapter{Sample queries}
            aaa

    \part{Conclusion}
        aaa

        \chapter{Final product}
            aaa

        \chapter{What I learned}
            aaa

        \chapter{Future}
            aaa

        \chapter{References}
            aaa

\end{document}
